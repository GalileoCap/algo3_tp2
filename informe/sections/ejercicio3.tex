\documentclass[../main.tex]{subfiles}

\begin{document}

\section{Ejercicio 3: Camino mínimo}

\subsection{Presentación}
\label{sec:ej3-intro}
\paragraph{} Este ejercicio presenta el problema \textit{UVa 1233, Uhser}\footnote{\url{https://onlinejudge.org/index.php?option=onlinejudge&Itemid=8&page=show_problem&problem=3674}}. En el que se quiere calcular el ciclo de costo mínimo.

\paragraph{} El problema plantea una iglesia donde los feligreses se pasan una caja donde ponen dinero, y un portero que al recibir la caja roba un dólar y la vuelve a pasar a algún feligrés. La forma en que se pasan la caja y cuánto dinero ponen está definido con unas reglas don de cada feligrés \(v\) le pasa la caja a algún otro feligrés o al portero \(w\) poniendo \(c_{v,w}\) dólares, y similarmente el portero tiene un conjunto de feligreses específicos a los que les puede pasar la caja. Una vez que la caja se llena con \(b\) dólares se pasa instantáneamente al sacerdote sin que el portero pueda robar más dólares. Se pide calcular la cantidad máxima de dólares que el portero puede llegar a robar.

\begin{figure}[H]
\centering

\begin{tikzpicture}
  \Vertex[x=0,y=1]{0}
  \Vertex[x=2,y=2]{1}
  \Vertex[x=2,y=0]{2}
  \tikzset{EdgeStyle/.style = {->}}
  \Edge[style={bend left=15}](0)(1)
  \Edge[style={bend left=15,red}](0)(2)
  \Edge[style={bend left=15},label=6](1)(0)
  \Edge[label=4](1)(2)
  \Edge[style={bend left=15,red},label=5](2)(0)
\end{tikzpicture}

\caption{Ejemplo de una iglesia representada en un digrafo, donde el 0 es el portero y los otros nodos feligreses y cada arista una regla. Con el ciclo mínimo marcado en rojo.}
\label{fig:ej3-ex}
\end{figure}

\subsection{Modelo y Algoritmo}
\label{sec:ej3-model}
\paragraph{} Modelamos el problema con un grafo dirigido con pesos donde cada feligrés está representado por un nodo, y cada arista es una regla que puede ejecutar. Agregamos un nodo \(v_{0}\) que representa al portero con aristas de peso 0 saliendo hacia cada feligrés al que le puede dar la caja.

\paragraph{} Va a haber una (o potencialmente varias equivalentes) secuencia de pasadas de la caja que maximiza las ganancias del portero, que en este grafo va a formar un ciclo de costo mínimo que incluye al portero. \\
Como el grafo no tiene ninguna arista de peso negativo, y por ende ningún ciclo negativo, utilizamos el algoritmo de camino mínimo de \textbf{Dijkstra} implementado con un \textbf{min-heap} para encontrar el camino mínimo desde el portero a cualquier feligrés en \(\bigO{|E|log(|V|)}\). \\
Luego conseguimos el ciclo mínimo revisando para cada feligrés que le puede devolver la caja al portero el costo de devolverle la caja sumado al costo del camino hacia ese feligrés. Esto se hace en \(\bigO{|V|}\). \\
Y finalmente calculamos la cantidad de dólares que el portero puede robar  con la siguiente ecuación, donde \(b\) es el tamaño de la caja y \(c_{min}\) es el costo del ciclo mínimo. La ecuación cuenta la cantidad de ciclos de peso mínimo en las que el portero se puede robar un dolar \(c_{min} - 1\) hasta que se junten los \(b\) dólares, y descuenta la última ronda en la que se alcanza el límite: \\
\[
  \Big\lceil\dfrac{b - c_{min}}{c_{min}-1}\Big\rceil
  \]
La complejidad total del algoritmo queda \(\bigO{|E|log(|V|) + |V|} = \bigO{|E|log(|V|)}\).

\subsection{Correctitud del algoritmo}
\label{sec:ej3-proof}
\paragraph{} Queremos determinar el monto máximo que el portero puede llegar a robar de la caja con capacidad $C$, de entre todos los posibles recorridos que existen entre el portero y los feligreses. El digrafo $G$ modela estas relaciones. \\
Sea $u$ el vértice que hace referencia al portero, de $u$ salen aristas de \textit{costo 0} hacia aquellos feligreses a los que puede pasarles la caja. Por otro lado, cada uno de los $n-1$ feligreses $f_i$ tendrá una arista incidente $e =$ ($f_i \rightarrow v$) con $c(e) = w_i$ y $v \in \lbrace u, f_1, ..., f_n \rbrace $. Con $w_i$ siendo la cantidad de monedas que tenga que poner respecto a la regla de dicha relación. La caja termina su recorrido una vez que su monto alcanza el valor $C$. Llegado este punto, el portero no podrá robar más monedas. \\
Buscamos que la caja pase por el portero la mayor cantidad de veces posible. Para lograr esto, se necesita que el monto que se acumula parcialmente en la caja, sea m\'inimo. Para ello, el portero necesita ser un vértice en G. % Buscamos que los recorridos que pueda realizar la caja, lo incluyan a él la mayor cantidad de veces. 
Podemos observar que la caja siempre empieza teniendo 0 monedas, asi que el portero es quien se la puede entregar a ciertos $f_i$ para que empiecen a llenarla. Entonces, ya tenemos como punto de partida a $u$. \\
% Luego, como buscamos que la caja vuelva al portero la mayor cantidad de veces, no sólo queremos encontrar aquellos $f_i'$ que lleguen a $u$, sino que pretendemos que lo hagan con el \textbf{menor costo posible}. 
Como la caja siempre sale de $u$, cuando la caja llegue a estos $f_i'$, y de alguno de ellos la caja regrese al portero, queda establecido un \textbf{circuito simple} que involucra a un subconjunto de feligreses y al portero. %En otras palabras, para que la caja regrese al portero, tiene que recorrer un circuito simple en G que incluya a u. 
Para dicho circuito simple, queremos que sea el de menor costo posible, para que pueda recorrerse la mayor cantidad de veces. \\
El algoritmo de \textbf{Dijkstra} devuelve todos los caminos de menor costo en un digrafo $G$, para un vértice particular. Este vértice queremos que sea $u$ porque justamente buscamos aquellos caminos mínimos hacia él, que permitan establecer un ciclo que lo incluya.

\end{document}
