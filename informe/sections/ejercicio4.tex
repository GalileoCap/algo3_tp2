\documentclass[../main.tex]{subfiles}

\begin{document}

\section{Ejercicio 4: Sistemas de restricciones de diferencias}

\subsection{Presentación}
\label{sec:ej4-intro}
\paragraph{} Este último ejercicio plantea resolver un sistema de ecuaciones utilizando el algoritmo de Fishburn.

\paragraph{} Se tiene un sistema \(\mathcal{S}\) de \(k\) ecuaciones sobre \(n\) variables \(x_{1}, \ldots, x_{n}\), donde cada ecuación es de la forma \(x_{i} - x_{j} \leq c_{i,j}\), y una lista \(\mathcal{D}\) de \(m\) números enteros ordenados de menor a mayor, y se quiere decidir si existe una asignación de las variables \(x_{i}\) a elementos en \(\mathcal{D}\) que satisfaga las ecuaciones del sistema.

\subsection{Algoritmo}
\label{sec:ej4-algo}
\paragraph{} El algoritmo es una modificación del algoritmo \textbf{Bellman-Ford}, que busca, en el contexto de este ejercicio, si existe una soluci\'on factible para el conjunto de desigualdades representadas en el grafo. Primero se le asigna el valor máximo posible \(D_{m}\) a cada variable en \(\bigO{n}\). Luego para cada inecuación se reduce a \(x_{j}\) lo menos posible para que satisfaga la inecuación en \(\bigO{k}\). Es decir, se le asigna a $x_j$ el elemento $k$ tal que $(D_k$ $\leq$ $x_i + c_{i,j}) \land (\nexists$ $D_r$ con $D_r$ $>$ $D_k$ $\land$ $D_r$ $\leq$ $x_i + c_{i,j})$. Notar que esta asignaci\'on es coherente ya que $D_k$ $\leq$ $x_i + c_{i,j}$ $\iff$ $D_k - x_i \leq c_{i,j}$ $\iff$ $x_j - x_i \leq c_{i,j}$, que es justamente lo que estamos buscando para cada inecuaci\'on. Esto lo repetimos hasta que una variable \(x_{j}\) se quede sin valores posibles o estén resueltas todas las inecuaciones. Finalmente, en base a todas las asignaciones de cada $x \in X$ con su respectivo elemento $k \in D$, si el conjunto $X$ efectivamente arroja una soluci\'on factible, entonces $X = \lbrace D_{k_1}, D_{k_2}, ..., D_{k_n} \rbrace$. \\
En peor caso se reduce a una sola inecuación en cada paso (recordemos que se tienen k inecuaciones), probando las \(m\) posibilidades para cada inecuación en \(\bigO{km}\) pasos. Como en cada caso se visitan a lo sumo todas las $n$ variables $x_i$, el algoritmo queda con un coste de \(\bigO{kmn}\).
\end{document}
