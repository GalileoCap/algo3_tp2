%&pdflatex
\documentclass[../main.tex]{subfiles}

\begin{document}

\section{Ejercicio 1: DFS}
\label{sec:ej1}

\subsection{Presentación}
\label{sec:ej1-intro}
\paragraph{} La primer consigna plantea resolver el problema \textit{UVa 12363, Hedge Mazes}\footnote{\url{https://onlinejudge.org/index.php?option=com_onlinejudge&Itemid=8&category=278&page=show_problem&problem=3785}}, en el que se busca contar todas las formas de resolver un laberinto que cumplan ciertas condiciones.

\paragraph{} El laberinto está formado por cuartos conectados entre sí con pasillos. Cada pasillo conecta dos cuartos y puede ser atravesado en ambas direcciones. \\
Los caminos que se buscan encontrar son caminos que arrancan y terminan en dos cuartos dados, y no pasan por un mismo cuarto más de una vez. En específico, se quiere revisar si hay un único camino que cumpla con esto.

\subsection{Modelado y Algoritmo}
\label{sec:ej1-modeling}
\paragraph{} Modelamos este problema con un grafo no dirigido \(G = (V, E)\), donde cada nodo representa un cuarto y cada arista un pasillo. Esto nos permite resolver el problema buscando los \textbf{caminos simples}, definidos como un camino dentro del grafo que no repite nodos, entre cualquier par de nodos, y revisando si son únicos.

\begin{figure}[H]
\centering

\begin{tikzpicture}
  \Vertex[x=2,y=3]{1}
  \Vertex[x=2,y=2]{2}
  \Vertex[x=3,y=1]{3}
  \Vertex[x=2,y=1]{4}
  \Vertex[x=1,y=1]{5}
  \Vertex[x=3,y=3]{6}
  \tikzstyle{LabelStyle}=[sloped]
  \tikzstyle{EdgeStyle}=[]
  \Edge[color=red](1)(2)
  \Edge[color=red](2)(3)
  \Edge[](2)(4)
  \Edge[](2)(5)
  \Edge[](4)(5)
\end{tikzpicture}
  
\caption{Ejemplo de un laberinto con el único camino simple resaltado en rojo.}
\label{fig:ej1-ex}
\end{figure}

\paragraph{} Si entre dos nodos hay más de un camino simple, se puede armar un ciclo uniéndolos. Por lo que nos interesa encontrar aquellos caminos simples armados por aristas \textbf{puente}, o sea, que no pertenezcan a ningún ciclo. Esto nos asegura que son únicos. \\
Averiguamos qué aristas son puente de manera eficiente haciendo \textbf{Depth First Search (DFS)} sobre el grafo, buscando aquellos tree-edges que no se hagan redundantes por un back-edge. La complejidad de hacer DFS es de \(\bigO{|V| + |E|}\). \\
Una vez encontradas todas las aristas puente, reconstruimos el grafo quedándonos sólo con estas aristas. Esto arma un subgrafo conformado por árboles donde dos nodos pertenecen a un mismo árbol sí y sólo si hay un único camino simple entre ellos. Esto lo hacemos nuevamente con DFS. \\
Finalmente hacemos DFS sobre este subgrafo marcando a cada nodo con la raíz de su árbol como su representante. Esto nos permite averiguar si dos nodos tienen un único camino entre ellos en \(\bigO{1}\), revisando si tienen misma raíz.

\begin{figure}[H]
\centering

\begin{tikzpicture}
  \Vertex[x=0,y=5,L=1(1)]{1}
  \Vertex[x=0,y=3,L=2(1)]{2}
  \Vertex[x=0,y=1,L=3(1)]{3}
  \Vertex[x=2,y=5,L=4(4)]{4}
  \Vertex[x=4,y=5,L=5(5)]{5}
  \Vertex[x=6,y=5,L=6(6)]{6}
  \Edge[](1)(2)
  \Edge[](2)(3)
\end{tikzpicture}
  
\caption{El grafo de la figura \ref{fig:ej1-ex} separado en los árboles con los representes entre paréntesis.}
\label{fig:ej1-res}
\end{figure}

\subsection{Correctitud}
\label{sec:ej1-proof}

\paragraph{} Para que el algoritmo es correcto probamos que en un grafo \(G = (V, E)\) se tiene un único camino simple \(C\) entre dos nodos \(s\) y \(t\) sí y solo sí \(C\) está compuesto por todas aristas puente.

\paragraph{\(\Rightarrow)\)} $"$Si \(C = \{s, \ldots, t\}\) es único, entonces \(C\) está compuesto por todas aristas puente.$"$ Lo probamos por el \textbf{absurdo}: \\
Asumimos que no está compuesto por todas aristas puente, esto significa que hay al menos una arista \(e = (v, w)\) tal que \(e\) no es puente y \(C = \{s, \ldots, v, w, \ldots, t\}\). \\
En ese caso en el subgrafo \(G' = G \backslash \{e\}\) hay un camino \(C' = \{s, \ldots, t\}\) que no usa a la arista \(e\), ya que, como \(e\) no es puente, en G' no se aumentan las componentes conexas, así que sigue existiendo un camino de s a t. Como \(e\) es parte de \(C\) se tiene que \(C \neq C'\). Que es absurdo porque \(C\) es único, por lo que todas las aristas de \(C\) son puente. \done
\paragraph{\(\Leftarrow)\)} $"$Si \(C = \{s, \ldots, t\}\) está compuesto por aristas puente, entonces \(C\) es el único camino simple entre \(s\)  y \(t\).$"$ Lo probamos por el \textbf{absurdo}: \\
Asumimos que hay otro camino simple \(C' = \{s, \ldots, t\}\) tal que \(C' \neq C\). Entonces se puede armar un ciclo que va de \(s\) a \(t\) por \(C\) y luego vuelve por \(C'\). Que es absurdo porque \(C\) está compuesto por todas aristas puente, por lo tanto, si se extrae de \(G\) una arista \(e = (v, w)\) tal que \(e\) $\in$ \(C\) $\land$ \(e\) $\notin$ \(C'\), no debería haber forma de ir de s a t, porque e era puente y por lo tanto, los vertices en $C_{sv}$ ya no pueden alcanzar a los vertices en $C_{wt}$. Esto se debe a que ahora pertenecen a componentes conexas distintas. Pero, si C' realmente existiera, entonces s y t seguirían siendo alcanzables porque \(e\) $\notin$ \(C'\), por lo que \(C'\) no se vería afectado. Absurdo, ya que \(e\) era puente entonces las componentes conexas aumentan al extraerlo de \(G\). Debido a esto, no existe ningún otro camino \(C' \neq C\) entre \(s\) y \(t\). \done

\paragraph{} Queda probado que un camino simple \(C = \{s, \ldots, t\}\) es único sí y solo sí \(C\) está compuesto por todas aristas puente. Por lo tanto, como nuestro algoritmo cuenta aquellos caminos formados íntegramente por aristas puente, cuenta todos los caminos simples del laberinto. Entonces, el algoritmo es correcto.

\end{document}
